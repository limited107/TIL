\documentclass[11pt]{oblivoir}

\begin{document}

\section{Unit 11-Present Perfect Continuous and Simple(I have been doing and I have doing)}

\subsection{A. Present Perfect Continuous와 Simple 비교}
\subsubsection{has been painting}
Ling's clothes are covered with paint. 
She \textbf{has been painting} the ceiling.
\textbf{Has been painting} is the \textit{present perfect continuous}

We are interested in the activity. It does not matter whether something has been finished or not. In this example, the activity (painting the ceiling) has not been finished.

\subsubsection{has painted}
The ceiling was white. Now it is red. She \textbf{has painted} the ceiling. \newline
\textbf{Has painted} is the \textit{present perfect simple}. 

Here, the important thing is that something has been finished. \textbf{Has painted} is a completed action. We are interested in the result of the activity (the painted ceiling), not the activity itself.

\subsubsection{Present Perfect Continuous(I have been doing)}
\begin{itemize}
  \item My hands are very dirty. I\textbf{'ve been fixing} the car.
  \item Joe \textbf{has been eating} too much recently.
  \item It's nice to see you again. What \textbf{have} you \textbf{been doing} since the last time we saw you?
  \item Where have you been? \textbf{Have} you \textbf{been} \textbf{playing} tennis?
\end{itemize}

\subsubsection{Present Perfect}
\begin{itemize}
  \item The car is OK again now. I\textbf{'ve fixed} it.
  \item Sombody \textbf{has eaten} all my candy. The box is empty.
  \item Where's the book I gave you? What \textbf{have} you \textbf{done} with it?
  \item \textbf{Have} you ever \textbf{played} tennis?
\end{itemize}

\subsection{continuous / simple와 함께 사용하는 어구}
\subsubsection{continous를 how long과 함께 사용}
계속 일어나고 활동을 말할 때 사용 

\begin{itemize}
  \item How long \textbf{have} you \textbf{been reading} that book?
  \item Lisa is still writing her report. She\textbf{'s been writing} it \textbf{all day}.
  \item They\textbf{'ve been playing} tennis \textbf{since 2:00}.
  \item I'm studying Spanish, but I \textbf{haven't been studying} it very long.
\end{itemize}

\subsubsection{simple을 how much, how many, how many times와 함께 사용}
완료된 행위를 말할 때 사용

\begin{itemize}
  \item How much of that book \textbf{have} you \textbf{read}?
  \item Lisa \textbf{has written} 10 pages today.
  \item They\textbf{'ve played} tennis three times this week.
  \item I'm studying Spanish, but I \textbf{haven't learned} very much yet.
\end{itemize}


\subsection{C. 예외}
\textbf{know}/\textbf{like}\textbf{believe}등은 continuous로 사용하지 않는다. (Unit 4A 참조)
\begin{itemize}
  \item I\textbf{'ve known} about it for a long time. (not I've been knowing)
\end{itemize}

want와 mean은 present perfect continuous를 사용할 수 있다.
\begin{itemize}
  \item {I \textbf{'ve been meaning} to phone Pat, but I keep forgetting. \newline Pat에게 전화하려고 했었는데, 나는 까먹고 있었어)
\end{itemize}

\end{document}
